%!TEX root = tikz.tex

\begin{frame}[fragile]{Artikelfassung}
  \begin{block}{Ziel}
    Generierung von Artikelfassung und Präsentation\\
    aus demselben Quellen-Dokument.
  \end{block}

  \xxx
  \pause

  \begin{alertblock}{Problem}
    \begin{tabular}{r@{ }l}
      \textbf{Präsentation} & Dokumentenklasse von \beamer.\\
      \textbf{Artikel} & Dokumentenklasse von KOMA-Script.
    \end{tabular}
  \end{alertblock}

  \xxx
  \pause

  \begin{block}{Lösung}
    \begin{itemize}
      \item Ein \LaTeX-Dokument für den Inhalt.
      \item Zwei \LaTeX-Dokumente für beide Dokumentenklassen.
      \item Einbinden des Inhalts mit \lstinline-\input-.
    \end{itemize}
  \end{block}
\end{frame}

\begin{frame}[fragile]{Einbinden des Inhalts}
  \begin{center}
    \begin{tikzpicture}[
        on grid,
        auto,
        node distance=18mm and 30mm,
        engine/.style={
          font=\rmfamily\Large\bfseries,
          inner sep=2pt
        }
      ]

      \node (content tex) {\shortstack{\textcolor{texicon}{\icon{TEX}}\\\texttt{content.tex}}};

      \uncover<2->{
        \node[below left=of content tex] (beamer tex)
          {\shortstack{\textcolor{texicon}{\icon{TEX}}\\\texttt{beamer.tex}}};
        \node[below right=of content tex] (article tex)
          {\shortstack{\textcolor{texicon}{\icon{TEX}}\\\texttt{article.tex}}};
      }

      \uncover<3->{
        \node[below=of beamer tex, engine]
          (make beamer) {\LaTeX.mk};

        \node[below=of article tex, engine]
          (make article) {\LaTeX.mk};

        \node[below=of make beamer] (beamer pdf)
          {\shortstack{\textcolor{pdficon}{\icon{PDF}}\\\texttt{beamer.pdf}}};
        \node[below=of make article] (article pdf)
          {\shortstack{\textcolor{pdficon}{\icon{PDF}}\\\texttt{article.pdf}}};
      }

      \only<2->{
        \draw[very thick]
        (content tex.west) edge[->] node[swap, near start]
          {\texttt{\color{texcs}\bfseries\textbackslash input}} (beamer tex.north);

        \draw[very thick]
          (content tex.east) edge[->] node[near start]
            {\texttt{\color{texcs}\bfseries\textbackslash input}} (article tex.north);
      }

      \only<3->{
        \draw[very thick]
          (beamer tex) edge[->] (make beamer)
          (article tex) edge[->] (make article);

        \draw[very thick]
          (make beamer) edge[->] (beamer pdf)
          (make article) edge[->] (article pdf);
      }
    \end{tikzpicture}
  \end{center}
\end{frame}

\begin{Frame}[fragile]{Inhalt \texttt{content.tex}}
  \begin{lstlisting}[gobble=4]
    % Präambel

    \title{Mein Vortrag}
    \author{Mein Name}

    \begin{document}
      \begin{frame}
        \maketitle
      \end{frame}

      \begin{frame}{Folientitel}
        Hier passierts \dots
      \end{frame}
    \end{document}
  \end{lstlisting}
\end{Frame}

\begin{frame}[fragile]{Dokumentenklassen}
  \inhead{Für die Folien \texttt{beamer.tex}}
  \begin{lstlisting}[gobble=4]
    % Beamer als Dokumentenklasse verwenden
    \documentclass{beamer}
    % gemeinsamen Inhalt einbinden
    \input{content.tex}
  \end{lstlisting}

  \xxx

  \inhead{Für den Artikel \texttt{article.tex}}
  \begin{lstlisting}[gobble=4]
    % KOMA-Script als Dokumentenklasse verwenden
    \documentclass{scrartcl}
    % Beamer als Paket laden
    \usepackage{beamerarticle}
    % gemeinsamen Inhalt einbinden
    \input{content.tex}
  \end{lstlisting}
\end{frame}

\begin{frame}[fragile]{Modes}
  \begin{tabular}{r@{ }l}
    \texttt{presentation} & nur für Folien\\
    \texttt{article} & nur für Artikel\\
    \texttt{all} & für Folien und Artikel (Standard)
  \end{tabular}

  \vskip3ex

  \begin{lstlisting}[gobble=4]
    \mode
    <name>
  \end{lstlisting}
  Wechselt den aktuellen Mode.

  \xxx

  \begin{lstlisting}[gobble=4]
    \mode*
  \end{lstlisting}
  Automatische Modeumschaltung:
  \begin{itemize}
    \item Innerhalb von \lstinline-frame- Mode {all}.
    \item Außerhalb von \lstinline-frame- Mode {article}.
  \end{itemize}
\end{frame}