%!TEX root = tikz.tex

\usepackage[utf8]{inputenc}
\usepackage[T1]{fontenc}

\usepackage{aurical}
\usepackage{mathptmx}
\usepackage[scaled]{helvet}
\usepackage{lmodern}
\usepackage{textcomp}
\usepackage[scaled=.88]{DejaVuSansMono}

\usepackage[ngerman]{babel}
\usepackage[german=guillemets]{csquotes}

\usepackage{amsmath}

\usepackage{tikz}
\usetikzlibrary{positioning,%
  fit,%
  arrows,%
  automata,%
  trees,%
  intersections,%
  mindmap,%
  shapes.geometric,%
  shapes.arrows,%
  decorations,%
  decorations.pathmorphing,%
  decorations.pathreplacing,%
  matrix,%
  chains,%
  scopes,%
  circuits,%
  circuits.ee.IEC,%
  calc,%
  fadings,%
  lindenmayersystems,%
  decorations.markings,%
  shadows.blur%
}

\usepackage{pifont}
\newcommand{\goodmark}{\textcolor{green!50!black}{\Pisymbol{pzd}{52}}}
\newcommand{\badmark}{\textcolor{red}{\Pisymbol{pzd}{56}}}

% define colors used by presentation theme AND article
\colorlet{maincolor}{orange}
\colorlet{alertedcolor}{red}
\colorlet{examplecolor}{green!50!black}

% color system
% - color!9 used as light fill color (e.g. for block content)
% - color!18 used as default fill color (e.g. for state)
% - color!30 used as highlight fill color (e.g. for block header)
% - color!50 used as strong highlight fill color (e.g. for header row)

\usepackage{listings}
\lstdefinelanguage[MyLaTeX]{TeX}[LaTeX]{TeX}%
  % TeX commands
  {moretexcs={enquote,includegraphics,%
    part,chapter,section,subsection,paragraph,subparagraph%
    tableofcontents,listoffigures,listoftables,maketitle,%
    subsection,subsubsection,paragraph,autoref,it,%
    textcolor,colorbox,xdefinecolor,colorlet,foreach,%
    rowcolors,rowcolor,lstdefinestyle,lstset,KOMAoptions,%
    setkomavar,setkomavar*,opening,closing,encl,%
    lehead,cehead,rehead,lefoot,cefoot,refoot,%
    lohead,cohead,rohead,lofoot,cofoot,rofoot,%
    ohead,chead,ihead,ofoot,cfoot,ifoot,%
    areaset,color,%
    automark,manualmark,markright,markboth,%
    tableofcontents,url,clearscrheadfoot,pagemark,headmark,%
    setheadtopline,setkomafont,setheadsepline,setfootsepline,%
    setfootbotline,chaptermark,thesection,thechapter,%
    thesubsection,subtitle,inst,section*,subsection*,institute,%
    chapter*,part*,%
    qedhere,usetheme,useinnertheme,useoutertheme,%
    pause,uncover,only,alert,onslide,mode,mode*,usetikzlibrary,%
    draw,filldraw,path,node,usefonttheme,setbeamertemplate,%
    declaretheorem,FiveFlowerOpen,%
    frontmatter,mainmatter,appendix,backmatter,%
    operatorname},%
  % LaTeX environments
  morekeywords={[2]lstlisting,document,letter,center,flushleft,%
    flushright,align,itemize,enumerate,description,tabular,%
    titlepage,figure,table,frame,tikzpicture,quote,quotation,verse},%
  % other things (like packages) to highlight
  morekeywords={[3]listings,textcomp,courier,xcolor,scrartcl,%
    scrlttr2,inputenc,babel,%
    fontenc,lmodern,mathptmx,%
    helvet,geometry,scrpage2,scrreprt,scrbook,%
    article,report,book,hyperref,%
    csquotes,amsmath,amssymb,la,beamer,beamerarticle,%
    tikz,amsthm,thmtools},
  alsoletter={0123456789*}
  }%

\lstdefinelanguage{BibTeX}
  {keywords={%
      @article,@book,@collectedbook,@conference,@electronic,@ieeetranbstctl,%
      @inbook,@incollectedbook,@incollection,@injournal,@inproceedings,%
      @manual,@mastersthesis,@misc,@patent,@periodical,@phdthesis,@preamble,%
      @proceedings,@standard,@string,@techreport,@unpublished%
      },
   comment=[l][\itshape]{@comment},
   morekeywords={[2]author,title,year,publisher,editor,booktitle,journal,series,volume,number,pages,institution,note,howpublished},
  }

\colorlet{texcs}{violet}
\colorlet{keyword}{violet}
\colorlet{keywordnd}{green!70!black}
\colorlet{keywordrd}{orange!70!black}
\colorlet{comment}{gray}

\lstset{%
  basicstyle=\ttfamily,%
  language=[MyLaTeX]TeX,%
  texcsstyle=*\color{texcs}\bfseries,%
  keywordstyle=\color{keyword}\bfseries,%
  keywordstyle={[2]\color{keywordnd}\bfseries},%
  keywordstyle={[3]\color{keywordrd}\bfseries},%
  commentstyle=\color{comment}\itshape,%
  stringstyle=\itshape,%
  numbers=none,%
  frame=lines,%
  backgroundcolor=\color{maincolor!10},%
  rulecolor=\color{maincolor!70},%
  framerule=1pt,%
  showstringspaces=false,%
  upquote=true,%
  framexleftmargin=3pt,%
  framexrightmargin=3pt}

\lstdefinestyle{pseudo}{language={},%
  basicstyle=\normalfont,%
  morecomment=[l]{//},%
  morekeywords={for,to,while,do,if,then,else},%
  mathescape=true,%
  columns=fullflexible}

\lstdefinestyle{block}{%
  backgroundcolor={},%
  frame=no,%
  aboveskip=0pt,%
  belowskip=0pt
}

% german umlauts and red braces
\lstset{
  literate={ö}{{\"o}}1
           {Ö}{{\"O}}1
           {ä}{{\"a}}1
           {Ä}{{\"A}}1
           {ü}{{\"u}}1
           {Ü}{{\"U}}1
           {ß}{{\ss}}1
           {\{}{{\textbf{\color{red}\{}}}1
           {\}}{{\textbf{\color{red}\}}}}1
           {[}{{\textbf{\color{red}[}}}1
           {]}{{\textbf{\color{red}]}}}1
           {<}{{\textbf{\color{red}<}}}1
           {>}{{\textbf{\color{red}>}}}1
}

% logos like \BibTeX
\usepackage{logos}

\usepackage{array}
\newcolumntype{L}[1]{>{\raggedright\let\newline\\\arraybackslash\hspace{0pt}}m{#1}}
\newcolumntype{C}[1]{>{\centering\let\newline\\\arraybackslash\hspace{0pt}}m{#1}}
\newcolumntype{R}[1]{>{\raggedleft\let\newline\\\arraybackslash\hspace{0pt}}m{#1}}

% zebra tables
\newcommand{\mainrowcolors}{\rowcolors{1}{maincolor!30}{maincolor!9}}
\newenvironment{zebratabular}{\mainrowcolors\begin{tabular}}{\end{tabular}}
\newcommand{\setrownumber}[1]{\global\rownum#1\relax}
\newcommand{\headerrow}{\rowcolor{maincolor!50}\setrownumber1}

% easier use of operatorname
\newcommand{\op}[1]{\operatorname{#1}}

\newcommand{\xxx}{\only<presentation>{\vskip1em}}

\newenvironment{mybib}{%
  \begin{thebibliography}{10}
}{%
  \end{thebibliography}
}

\makeatletter
% Reverts "Heiko's fix for correct generation of section* and
% subsection* bookmarks" in beamerbasecompatibility.sty which
% disables bookmarks in \section* and \subsection*.
% This is done by disabling \Hy@writebookmark in \beamer@section
% and \beamer@subsection if their second argument is empty.
% This is the case for \section* and \subsection* as these
% commands are mapped to \beamer@section[{#1}]{} resp.
% \beamer@subsection[{#1}]{}. By defining \beamer@section and
% \beamer@subsection again to the original and not patched version
% in the macro \beamer@lastminutepatches where the patch is applied
% this behaviour gets disabled. This is mainly reverted because
% bookmarks of parts are not added after \section*{Conclusion} due
% to this patch.
\g@addto@macro\beamer@lastminutepatches{
  \ifx\org@beamer@section\undefined{}%
  \else
    \let\beamer@section\org@beamer@section{}%
  \fi
  \ifx\org@beamer@subsection\undefined{}%
  \else
    \let\beamer@subsection\org@beamer@subsection{}%
  \fi
}
\makeatother

% display math in serif font on the slides
\usefonttheme[onlymath]{serif}

\newcommand{\inhead}[1]{\textbf{\color{maincolor}#1}}

\useinnertheme{default}
\usefonttheme{structurebold}
\setbeamertemplate{navigation symbols}{} % hide navigation symbols

\setbeamercolor{structure}{fg=maincolor}

\setbeamercolor{block title}{fg=black,bg=maincolor!30}
\setbeamercolor{block title alerted}{fg=black,bg=alertedcolor!30}
\setbeamercolor{block title example}{fg=black,bg=examplecolor!30}

\setbeamercolor{block body}{parent=normal text,use=block title,bg=maincolor!9}
\setbeamercolor{block body alerted}{parent=normal text,use=block title,bg=alertedcolor!9}
\setbeamercolor{block body example}{parent=normal text,use=block title,bg=examplecolor!9}

\setbeamercolor{page number in head/foot}{fg=black}

\setbeamercolor{alerted text}{fg=alertedcolor}
\setbeamercolor{example text}{fg=examplecolor}

\pgfdeclareimage[width=8cm]{live-coding}{live-coding}
\newcommand{\livecoding}[1]{\frame{\frametitle{#1}\centerline{\pgfuseimage{live-coding}}}}

% create new environment Frame behaving exactly like frame
% useful for having listings with \end{frame}
\newenvironment{Frame}[1][]{%
  \begin{frame}[environment=Frame,#1]}{\end{frame}}
